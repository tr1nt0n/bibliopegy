\documentclass[12pt]{article}
\usepackage{fontspec}
\usepackage[utf8]{inputenc}
\setmainfont{Bodoni 72 Book}
\usepackage[paperwidth=17in,paperheight=11in,margin=1in,headheight=0.0in,footskip=0.5in,includehead,includefoot,portrait]{geometry}
\usepackage[absolute]{textpos}
\TPGrid[0.5in, 0.25in]{23}{24}
\parindent=0pt
\parskip=12pt
\usepackage{nopageno}
\usepackage{graphicx}
\graphicspath{ {./images/} }
\usepackage{amsmath}
\usepackage{hyperref}
\usepackage{tikz}
\newcommand*\circled[1]{\tikz[baseline=(char.base)]{
            \node[shape=circle,draw,inner sep=1pt] (char) {#1};}}

\begin{document}

\vspace*{1\baselineskip}

\begingroup
\begin{center}
\huge NOTES FOR THE INTERPRETERS
\end{center}
\endgroup

\begingroup
\textbf{Staging: \circled{1}} This piece uses a \textbf{pseudo-antiphonal staging} wherein the viola soloist is at the \textbf{center} of a rectangle of \textbf{four chamber ensembles.} These chamber ensembles, and their orientation to the violist starting at the front left corner and moving clockwise are: \circled{1} Flute, bass flute, violin; \circled{2} cello trio; \circled{3} tenor trombone, bass trombone, percussion ii; \circled{4} bass clarinet, percussion i. \textbf{\circled{2} If multiple levels of elevation are available}, it is preferable that chamber ensembles \textbf{1}, \textbf{2}, and \textbf{4} are elevated \textbf{above} the violist, and chamber ensemble \textbf{3} is positioned \textbf{below} the violist, while maintaining their horizontal rectangular orientation. \textbf{\circled{3} To facilitate simultaneous playing}, the ensemble may choose to use \textbf{synchronised stopwatches}, following the time stamps above the staff, or to video broadcast a \textbf{conductor} to each group, similar to the common practice found in opera production. \textbf{\circled{4}} In light of these considerations, it is \textit{highly} recommended that this piece be played \textbf{from the full score}, although parts for each chamber ensemble are provided. 
\endgroup

\begingroup
\textbf{General: \circled{1} Space notation} is used simultaneously with the rhythmic notation developed within the Western classical music tradition in this score. In the absence of a \textbf{time signature}, ``un-rhythmed" note heads are to played within the space of their \textbf{one-second-long measure}. In this idiom, \textbf{sustained rhythms} are indicated with a \textbf{straight line} emanating from the relevant note. \textbf{\circled{2}} After temporary \textbf{accidentals}, cancellation marks are printed also in the following measure ( for notes in the same octave ) and, in the same measure, for notes in other octaves, but they are printed again if the same note appears later in the same measure, except if the note is immediately repeated. \textbf{\circled{3} Microtones} in this score are \textbf{quarter-tones}, \textbf{eighth-tones}, and \textbf{cents}. An \textbf{inverted flat symbol} indicates a \textbf{quarter-tone flat}, while a \textbf{sharp symbol with only one vertical line} indicates a \textbf{quarter-tone sharp}. Any accidental can be modified with a \textbf{downwards- or upwards-facing arrow} to indicate an \textbf{eighth-tone flat or sharp}. \textbf{\circled{4} Justly tuned intervals} are indicated by the use of \textbf{Helmholtz-Ellis accidental system} combined with \textbf{cent deviations from equal temperament} for use with an electronic tuner. When no example pitch is given with the cent deviation, the mark is a deviation of the nearest ``standard” accidental. In the absence of electronic tuners, approximations of these deviations are acceptable. When Helmholtz-Ellis notation is not given, the pitches are to be played as usual. \textbf{\circled{5} Playing techniques} apply only to the note to which they are attached. If a technique is to persist for longer than a single note, a hooked, dashed line will span the music as long as the technique is active. \textbf{\circled{6} Arrows above the staff} indicate a gradual transition from one technique to another. \textbf{\circled{7} Triangular note heads} facing \textbf{upwards} or \textbf{downwards} indicate to play the \textbf{lowest} or \textbf{highest possible note}, respectively. ``Lowest" and ``highest" are just as reliant on factors such as dynamic and technique as they are on instrumentation. \textbf{\circled{8} Trills} are always between the \textbf{fundamental} and the pitch \textbf{one major second higher}, unless otherwise specified. \textbf{The width of trills} may vary, indicating a free \textbf{accelerando} or \textbf{ritardando}. \textbf{\circled{9} Dashed slurs} indicate legato playing without prescribing bowing or tonguing. \textbf{\circled{10} Instrument changes} ( especially relevant to the \textbf{violist}, \textbf{flutists}, and \textbf{percussionists} ) are signaled by \textbf{boxed text} containing the name of the \textbf{instrument to be switched to}. \textbf{\circled{11} Every musician} should be equipped with \textbf{two Baoding Balls} {\setmainfont{Source Han Serif SC}\selectfont \textbf{( 保定健身球 )}}.
\endgroup

\begingroup
\textbf{Viola: \circled{1} The viola is amplified}, preferably using two contact microphones. \textbf{\circled{2} The viola is secured to a table}, around which are \textbf{4 loudspeakers} playing the \textbf{fixed media}, and \textbf{amplification} and \textbf{electronic processing} of the viola's microphone signal. \textbf{\circled{3} The viola is prepared} with \textbf{styrofoam} between the \textbf{bridge} and \textbf{strings II and III}. \textbf{\circled{4} String II} of the viola is \textbf{detuned} to \textbf{A-quarter-sharp 3}. \textbf{\circled{5} Other materials} which the violist should have available are: \circled{1} Two bows; \circled{2} a piece of styrofoam, secured to the edge of the table to be bowed. \textbf{\circled{6} Live processing} of the viola is accomplished using supercollider. The code required for this piece can be found at this link: \\
\url{https://github.com/tr1nt0n/bibliopegy/blob/main/bibliopegy/sc/viola_processing.scd}. \\
\textbf{Four} effects labelled using \textbf{Hanzi numerals} may be activated and deactivated according to the score by either the violist or an engineer. \textbf{\circled{7}} Unless playing with two bows, it is preferred that the entire piece be played with \textbf{two hands on the bow}, one at \textbf{au talon}, and one at \textbf{punta d`arco}. \textbf{\circled{8} Tuning peg glissandi} on strings \textbf{I and IV} appear frequently throughout the piece. They are always \textbf{double stops}, so a tuning peg glissando on string \textbf{IV} will always be a dyad including \textbf{G 3}, and a tuning peg glissando on string \textbf{I} will always be a dyad including \textbf{A quarter-sharp 3}. \textbf{\circled{9} Degrees} above the staff indicate the \textbf{angle of the bow}, wherein \textbf{+45°} indicates to point the tip of the bow \textbf{as far upward as possible}, \textbf{0°} indicates a bow \textbf{completely perpendicular} to the instrument, and  \textbf{-45°} indicates to point the tip of the bow \textbf{as far downward as possible}. \textbf{\circled{10} A three line staff with a bridge clef} indicates to drag the bow \textbf{vertically} across the strings. In this staff, the top line indicates \textbf{the tailpiece}, the second indicates \textbf{the bridge}, and the lowest line indicates \textbf{halfway up the fingerboard}. All \textbf{points between theses lines} are \textbf{approximate} positions between the fixed positions of the lines. If a \textbf{zigzagged glissando} is used with this staff, it indicates to press the bow against the strings until the hair of the bow touches the wood of the bow and twist the wood against the hair, while maintaining the prescribed string contact point. \textbf{\circled{11}} From minute 2'20" of the movement \textit{Desiderata}, a \textbf{Baoding Ball} is used to close the strings rather than the left hand. This is signaled in the score as ``\textbf{Baoding Ball Glissando}."
\endgroup

\break

\vspace*{1\baselineskip}

\begingroup
\textbf{Flutes: \circled{1}} Each flutist should also be equipped with a \textbf{piccolo}.  \textbf{\circled{2} When instructed to cover the entire mouthpiece with the lips}, the lips should still be \textbf{held together}, not so that they buzz against each other, but so that the sound of the air passing through the lips is amplified by the body of the flute. \textbf{\circled{3} The international phonetic alphabet} is sometimes used to indicate \textbf{simultaneous soundings of the mouth}. \textbf{\circled{4} Square note heads} indicate \textbf{aeolian sound}. \textbf{\circled{5} Tremoli} always indicate \textbf{fluttertongue}. \textbf{\circled{6} The grace notes on the beat} from minute 2'20" of the movement \textit{Desiderata} onwards indicate \textbf{overblowing through the partials of a fundamental}. This gesture should be played quickly and explosively, nevertheless beautifully, with the fundamental being held for the remainder of the relevant note's duration.
\endgroup

\begingroup
\textbf{Strings ( including Viola ): \circled{1}} The \textbf{abbreviations} used in this score are: \circled{1} \textbf{Pont.} for \textbf{sul ponticello}; \circled{2} \textbf{tast.} for \textbf{sul tasto}; \circled{3} \textbf{dietro pont.} for \textbf{playing on the strings between the bridge and the tailpiece}; \circled{4} \textbf{ord.} for \textbf{ordinario}; \circled{5} \textbf{norm.} for \textbf{normale}; \circled{6} and \textbf{scratch} or \textbf{scr.} for \textbf{scratch tone}. \textbf{\circled{2}} Three degrees of \textbf{finger pressure} are used in this piece: \textbf{Fully closed string}, signified with a \textbf{standard note head}; \textbf{harmonic finger pressure}, signified with a \textbf{white, diamond-shaped note head}; and \textbf{half-pressure}, signified with a \textbf{half-open diamond note head}; played between a harmonic and a fully closed string. These degrees of pressure may be interpolated between, using an arrow above the staff between two relevant note head symbols. It should be noted that there are \textbf{no artificial harmonics} in this piece, so a harmonic note head above a standard note head always should be interpreted as a \textbf{double stop}. \textbf{\circled{3}} In various passages throughout this piece, there is notation which represents \textbf{the point at which the bow is touched} as it is drawn across the string. These positions are written as \textbf{fractions} where \textbf{0/7} and \textbf{0/5} represent \textbf{au talon} and \textbf{7/7} and \textbf{5/5} represent \textbf{punta d`arco}. For the duration of the note to which these fractions are attached, the interpreter should draw the bow at a constant speed, moving toward the destination point indicated on the following note. Bowings are provided. Passages without these indications should be bowed at the interpreter’s discretion. In the celli, these fractions are sometimes attached to \textbf{red grace notes on the beat}. In this case, the rhythm of the bow contact interpolation should be interpreted freely based on the duration of the grace notes.
\endgroup

\begingroup
\textbf{Bass clarinet: \circled{1} Multiphonics} are notated with the a \textbf{finger chart above the fundamental}. Neither the fundamental nor all possible overtones of the multiphonic must sound. \textbf{\circled{2}} All \textbf{trills} should be interpreted as \textbf{bisbigliando timbre trills}. \textbf{\circled{3} Rhythmed timbre alterations} are notated as a circled number above a note ( such as \circled{1}, \circled{2}, or \circled{3} ), where \textbf{higher numbers} refer to a \textbf{greater deviation in timbre and pitch}.
\endgroup

\begingroup
\textbf{Trombones: \circled{1}} Each trombonist should be equipped with \circled{1} \textbf{a vinyl cover}, preferably a vinyl record large enough to cover the bell of the instrument, which should be held lightly against the bell to produce a buzzing when directed, and \circled{2} a bassoon mouthpiece, to be exchanged with the trombone's mouthpiece when directed. \textbf{\circled{2}} When playing with the bassoon mouthpiece, the interpreters occasionally read a \textbf{two line staff} wherein the \textbf{top line} indicates \textbf{the first slide position}, the \textbf{bottom line} indicates \textbf{the seventh slide position}, and \textbf{the space between} indicates \textbf{approximate positions between the two}. 
\endgroup

\begingroup
\textbf{Percussions: \circled{1} The instruments} and \textbf{corresponding implements} of the \textbf{first percussionist} are as follows: \\
\circled{1} \textbf{A timpani}, prepared with a crash cymbal placed upside-down on the head, played with \textbf{two hard timpani mallets}. \\
\circled{2} \textbf{A snare drum}, prepared with a crash cymbal placed right-side-up on the head, to be used as the bridge for a piece of fishing line, which can be drawn across the drum and bowed. Played with \textbf{two bows}, or the above \textbf{timpani mallets}. \\
\circled{3} \textbf{A large anvil} and a \textbf{small anvil}, each played with a \textbf{hollow aluminum pipe}. \\
\circled{4} \textbf{A mounted thunder tube} to be \textbf{bowed} on the \textbf{string}. \\
\endgroup

\begingroup
\textbf{\circled{2} The instruments} and \textbf{corresponding implements} of the \textbf{second percussionist} are as follows: \\
\circled{1} \textbf{A five-octave marimba}, played with \textbf{two rubber mallets}. \\
\circled{2} \textbf{A glockenspiel}, played with \textbf{two plastic mallets}. \\
\endgroup

\begingroup
\textbf{\circled{3} When the first percussionist is playing the anvils}, a sustained note is accomplished by \textbf{scraping}, and a short one by \textbf{striking}.
\textbf{\circled{4} Grace notes on the beat} should be played \textbf{as quickly as possible}, afterwards returning to the \textbf{fundamental note}.
\textbf{\circled{5} A zigzagged glissando} indicates \textbf{a chromatic scale} from one note to another.  
\endgroup

\end{document}